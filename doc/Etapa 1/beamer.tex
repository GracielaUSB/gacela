\documentclass{beamer}

\mode<presentation>
{
  \usetheme{Warsaw}
  \usecolortheme{default}
  \usefonttheme{default}
  \setbeamertemplate{navigation symbols}{}
  \setbeamertemplate{caption}[numbered]
}

\usepackage[spanish]{babel}
\usepackage[utf8x]{inputenc}
\usepackage[T1]{fontenc}
\usepackage{lmodern}
\usepackage{syntax}

\setlength{\grammarparsep}{5pt plus 1pt minus 1pt}
\setlength{\grammarindent}{7em}

\title[Graciela]{[Gg]raciela - Primera entrega}
\author[Ackerman - Spaggiari]{Moisés Ackerman \and Carlos Spaggiari}
\institute[USB]{Universidad Simón Bolívar}
\date{\today{}}

\begin{document}

\begin{frame}
  \titlepage
\end{frame}


\section{Lo superficial}

\subsection{Nombres}
\begin{frame}{Nombres}
\begin{itemize}
  \item This is not \textit{gacela}.
  \item El lenguaje es Graciela.
  \item El compilador es \texttt{graciela}.
\end{itemize}
\end{frame}


\subsection{El comando}
\begin{frame}{El comando}
\framesubtitle{UX stands for stUdent eXperience}
\begin{itemize}
  \item Compilar antes: \begin{itemize}
    \item\texttt{gacela tarea.gcl 0}
    \item\texttt{llc-3.5 -filetype=obj tarea.bc}
    \item\texttt{gcc tarea.o /lib/x86\_64-linux-gnu/auxiliarFunctions.so -o tarea}
  \end{itemize}
   \item Compilar ahora: \begin{itemize}
    \item\texttt{graciela tarea.gcl -o tarea}
    \item That's it.
    \item Si el estudiante quiere ver más o menos errores, flag \texttt{-e=n}.
  \end{itemize}
\end{itemize}
\end{frame}


\subsection{Instalación}
\begin{frame}{Instalación}
\framesubtitle{UX stands for stUdent eXperience}
\begin{itemize}
  \item apt-get (Ubuntu): \begin{itemize}
    \item\texttt{\# add-apt-repository ppa:graciela-usb/ppa}
    \item\texttt{\# apt-get install graciela}
  \end{itemize}
  \item Homebrew (OS X) : \begin{itemize}
    \item\texttt{\$ brew tap GracielaUSB/graciela}
    \item\texttt{\$ brew install graciela}
  \end{itemize}
\end{itemize}
\end{frame}


\subsection{Add-ons}
\begin{frame}{Colores!}
\framesubtitle{UX stands for stUdent eXperience}
\begin{itemize}
  \item\visible<1->{Resaltador de sintaxis para Sublime Text},
  \item\visible<2->{próximamente para otros editores}.

  \vskip 1cm

  \item\visible<3->{*breve demostración*}
\end{itemize}
\end{frame}


\subsection{Fixes}
\begin{frame}{Fixes}
\framesubtitle{Look ma, no floats!}
\begin{itemize}
  \item\visible<1->{El compilador no muestra (tantos) errores de Haskell}
  \item\visible<2->{Se mejoró la implementación de la recuperación de errores en el analizador sintáctico}
  \item\visible<3->{Los flotantes no dan error de compilación}
\end{itemize}
\end{frame}


\subsection{Pendientes}
\begin{frame}{Pendientes}
\framesubtitle{\texttt{Non-exhaustive patterns in function <insert function here>}}
\begin{itemize}
  \item Sigue habiendo errores en la generación de código para cuantificadores (rangos),
\end{itemize}
\end{frame}


\section{Lo profundo}
\subsection{Objetivos}
\begin{frame}{Objetivos}
\begin{itemize}
  \item Tipos de datos estructurados abstractos
  \item Tipos de datos estructurados
  \item Apuntadores
\end{itemize}
\end{frame}


\subsection{Tipos de datos estructurados abstractos}
\begin{frame}[fragile]{La gramática}
\scriptsize
\begin{grammar}

<ADT> ::= `abstract' <Id> <TypeVars> `begin' <AbstBody> `end'

<TypeVars> ::= `(' <Id> <MoreIds> `)' | $\lambda$

<MoreIds> ::= `,' <Id> <MoreIds> | $\lambda$

<AbstBody> ::= <AbstVarDecs> <Invariant> <ProcDecs>

<AbstVarDecs> ::= <AbstVarDec> `;' <AbstVarDecs> | $\lambda$

<AbstVarDec> ::= (`var'|`const') <Id> <MoreIds> `:' <AbstType>

<AbstType> ::= set of <Basic'>
\alt multiset of <Basic'>
\alt seq of <Basic'>
\alt Func <Basic'> `->' <Basic'>
\alt Rel <Basic'> `->' <Basic'>
\alt `(' <Basic'> `,' <Basic'> <MoreBasics'> `)'
\alt <Basic'>

<Basic'> ::= <Id> | <Basic>

<Basic> ::= `int' | `char' | `double' | `bool'

\end{grammar}
\end{frame}

\begin{frame}[fragile]{La gramática, cont.}
\scriptsize
\begin{grammar}

<Invariant> ::= `\{inv' <BoolExp> `inv\}'

<ProcDecs> ::= <ProcDec> <ProcDecs>

<ProcDec> ::= `proc' <Id> `(' <ProcArgs> `)' <Precond> <Postcond>

<ProcArgs> ::= <ProcArg> `,' <ProcArgs> | $\lambda$

<ProcArg> ::= (`in'|`out'|`inout'|`ref') <Id> `:' <AbstType'>

<AbstType'> ::= <AbstType>
\alt <Id> <TypeVars>

<Precond> ::= `\{pre' <BoolExp> `pre\}'

<Postcond> ::= `\{post' <BoolExp> `post\}'

\end{grammar}
\end{frame}


\subsection{Tipos de datos estructurados}
\begin{frame}[fragile]{La gramática}
\scriptsize
\begin{grammar}

<DT> ::= `type' <Id> <Types> `begin' <DTBody> `end'


<Types> ::= `(' <Type> <MoreTypes> `)' | $\lambda$

<Type> ::= array of <Type>
\alt `*' <Type>
\alt <Basic>

<MoreTypes> ::= `,' <Type> <MoreTypes> | $\lambda$

<Body> ::= <VarDecs> <Repinv> <Coupinv> <ProcDefs>

<Repinv> ::= `\{repinv' <BoolExp> `repinv\}'

<Coupinv> ::= `\{coupinv' <BoolExp> `coupinv\}'

<ProcDefs> ::= <ProcDef> <ProcDefs>

<ProcDef> ::= \textit{(Definido por Araujo y Jiménez)}

\end{grammar}
\end{frame}


\subsection{Acoplamiento}
\begin{frame}{Verificación de precondiciones}
\begin{enumerate}
\item Se verifica la precondición del TD (concreto). Si falla, se da una
      \textbf{\textit{advertencia}} a tiempo de ejecución, ``No se cumplió la precondición'',
      y no se hacen más verificaciones dentro de esta llamada. Si se cumple,
      la verificación sigue.

\item Se utiliza el invariante de acoplamiento para generar las estructuras
      del TDA a partir de las del TD (concreto).

\item Se verifica la precondición del TDA con las estructuras generadas
      en el paso anterior. Si falla, se da un \textbf{\textit{error}} a tiempo de
      ejecución, ``Este procedimiento no implementa el abstracto''. Si se cumple,
      se ejecuta el cuerpo del método.
\end{enumerate}
\end{frame}


\begin{frame}{Verificación de poscondiciones}
\begin{enumerate}
\item Se utiliza el invariante de acoplamiento para generar las estructuras
      del TDA a partir de las del TD (concreto).

\item Se verifican tanto la poscondición del TD (concreto) como la del TDA,
      existiendo las siguientes posibilidades:

\vskip 0.5cm

\tiny\centering
\begin{tabular}{| l || c | c |}
  \hline
  Poscondición \ldots & abstracta se cumple & abstracta falla \\
  \hline \hline
  concreta se cumple & \parbox[t]{2cm}{Éxito} & \parbox[t]{3cm}{``Este procedimiento no implementa el abstracto, revisar acoplamiento''} \\
  \hline
  concreta falla & ``No cumple poscondición'' & ← y ↑ \\
  \hline
\end{tabular}

\end{enumerate}
\end{frame}


\subsection{Apuntadores}
\begin{frame}{Apuntadores}
\begin{itemize}
  \item Se desea un uso sencillo pero sin límites excesivos para los apuntadores
  \item Se considera apropiado manejar apuntadores exclusivamente a estructuras en el \textit{heap},
    manejados con palabras clave \texttt{new} y \texttt{free}.
  \item Interacción delicada con tipos de datos estructurados al momento de verificar aserciones.
\end{itemize}
\end{frame}


\section{Resultados}

\subsection{Resultados esperados}
\begin{frame}{Resultados esperados}
\begin{itemize}
  \item Especificación formal de la extensión de Graciela a implantar.
  \item Extensión del front-end del compilador.
\end{itemize}

\end{frame}

\subsection{Actividades propuestas}
\begin{frame}{Actividades propuestas}
\begin{itemize}
  \item Evaluar las recomendaciones del jurado del proyecto de grado de Araujo y
Jiménez sobre la semántica del lenguaje propuesta por ellos.
  \item Revisión de la bibliografía existente para el manejo de tipos definidos
por el usuario en el contexto de programación formal.
  \item Implantación de un Lexer y un Parser extendidos que soporten las nuevas
funcionalidades propuestas.
\end{itemize}

\end{frame}

\end{document}
