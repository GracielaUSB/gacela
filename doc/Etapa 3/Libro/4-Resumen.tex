\renewcommand{\thepage}{\Roman{page}}

\setcounter{page}{4} 

\begin{center}
\textbf{\large Resumen}
\par\end{center}{\large \par}

\vspace{5mm}


El presente proyecto de pasantía consiste en el diseño y la implementación
de un componente publicitario para ser incluído en las aplicaciones
desarrolladas para los sistemas operativos móviles iOS y Android.
La empresa Mobile Media Networks C.A. tiene como objetivo ampliar
su alcance publicitario correspondiente a la colocacion de campañas
en aplicaciones móviles. Gracias al éxito de ventas de campañas publicitarias
con la plataforma Blackberry, se desea llegar a los usuarios de otras
tecnologías que están siendo cada vez más populares en Venezuela.
Estas son iOS por la empresa Apple y Android de la empresa Google.
Por lo tanto es necesario la creación de un componente nativo por
cada plataforma que permita incluir de manera sencilla la publicidad
en las aplicaciones.

El componente actual, creado para aplicaciones de Blackberry, se comunica
con el servidor mediante un protocolo que genera una respuesta muy
pesada con información inutil para la colocación de la piblicidad.
Por lo tanto el primer paso es el diseño de un nuevo protocolo que
produzca una respuesta de menor tamaño y con la información mínima
necesaria, lo que tendrá un impacto favorable tanto en espacio como
en tiempo. Una vez tomada la decisión sobre la comunicación con el
servidor, se realizará el diseño de los componentes, analizando el
actual y considerando las posibles mejoras. El componente trabaja
en conjunto con el AdServer MobileX, un servidor que se encarga de
enviar la publicidad de manera inteligente. La realización de los
cambios en el servidor para implementar la nueva comunicación no está
en el alcance de este proyecto.

El componente debe ser capaz de colocar en cualquier lugar de la pantalla
del dispositivo la publicidad, sin importar el tipo de la imagen o
su tamaño original. Además debe reportar al servidor cada vez que
se mostró o que el usuario hizo clic sobre ella. La entrega oportuna
de esta información es crucial para la venta apropiada de campañas
a los clientes y la generación de reportes de las mismas. Finalmente,
debe manejar problemas como fallas en la conexión, el almacenamiento
o la posible sobrecarga del servidor, así como algunas necesidades
nuevas que permiten segmentar la publicidad por país o por tipo de
usuario. 

\begin{onehalfspace}
La metodología utilizada para la implementación de este proyecto es
AUP, una metodología de tipo ágil, con un buen balance entre la planificación
del proyecto y tiempos de desarrollo del software.\newpage{}\end{onehalfspace}

