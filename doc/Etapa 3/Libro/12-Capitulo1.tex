
\chapter{Entorno Empresarial}

\thispagestyle{empty} 

Este capítulo da a conocer el entorno empresarial en el cual se desarrolla
el proyecto de pasantías. Busca mostrar la labor de Mobile Media Networks,
dando una idea de sus objetivos, misión y aspectos históricos relevantes. 


\section{Mobile Media Networks}

La empresa Mobile Media Networks se dedica al desarrollo de portales
y aplicaciones en dispositivos móviles, así como la distribución de
publicidad en los mismos. A través del manejo de distintas plataformas
móviles, diseñan y programan aplicaciones que logran integrar el contenido
y la publicidad de distintas empresas a tabletas, celulares y otros
dispositivos portatiles. A continuación se presentará una breve descripción
de la empresa.


\subsection{Reseña Histórica}

Mobile Media Networks nace de la empresa \textquotedblleft{}Adsmedia
Mobile Advertising\textquotedblright{} de España, la cual se dedica
al mercadeo móvil y a la publicidad. Es en Barcelona, en el año 2006,
cuando es fundada Adsmedia Mobile Advertising, como una empresa global,
joven y dinámica con la capacidad de adaptar de forma inmediata la
oferta de productos y servicios publicitarios a las necesidades nuevas
y cambiantes del mercado. 

En el año 2007 gana el premio \textquotedblleft{}Empresa Innovadora\textquotedblright{}
de la Generalitat de Catalunya, el premio \textquotedblleft{}las 100
mejores ideas empresariales\textquotedblright{} otorgado por la revista
Actualidad Económica y queda como finalista en BCN-Emprendedores edición
2007. Al poco tiempo es seleccionada por Nokia como \textquotedblleft{}partner\textquotedblright{}
estratégico para el desarrollo de Publicidad Móvil en España, Portugal
y Latinoamérica. En mayo de este año se abren nuevas oficinas en Londres
y en Caracas.\cite{Cap1.Hitos}

En el 2008 surgen nuevos contratos de gestión y desarrollo con empresas
como Movistar, Vodafone, Orange, Meridiano y Digitel. También tienen
lugar expansiones y nuevas versiones del sistema encargado de la distribución
de publicidad, \textquotedblleft{}Mobile Advertising Adserver\textquotedblright{},
ofreciendo nuevas tecnologías como el soporte de videos. También se
realiza la apertura de oficinas en Brasil. 

En el año siguiente es desarrollada y puesta en uso la nueva versión
del sistema \textquotedblleft{}Mobile Advertising Adserver\textquotedblright{}
y nuevas oficinas son inauguradas en Bahrein y Lisboa. Por otro lado
la empresa tiene presencia en el Mobile World Congress 2009. \cite{Cap1.Hitos}

A mediados del año 2009 ocurre el traspaso de las operaciones de \textquotedblleft{}Adsmedia
Mobile Advertising\textquotedblright{} en España a \textquotedblleft{}Adsmedia
Mobile Advertising\textquotedblright{} en Venezuela, otorgando a estas
oficinas la gestión del conjunto de sucursales distribuidas a nivel
mundial y el control de su desarrollo. Poco después esta empresa adquiere
independencia con respecto a la sucursal en España y en el año 2010
cambian su nombre a \textquotedblleft{}Mobile Media Networks\textquotedblright{}. 

Actualmente Mobile Media Networks posee oficinas en Caracas, Bogotá
y Miami, prestando los servicios de distribución de publicidad en
portales web y dispositivos portátiles, así como el desarrollo de
aplicaciones móviles para las plataformas iOS, Android y Blackberry.


\subsection{Misión}

Mobile Media Networks es la empresa especializada en brindar soluciones
de mercadeo en celulares y publicidad en Latinoamérica y la ciudad
de Miami.


\subsection{Visión}

Mobile Media Networks presenta su visión como: 

\textquotedblleft{}Creer en la liberalización de las comunicaciones
y en el desarrollo de la publicidad en el móvil. El móvil se ha convertido
en un nuevo soporte publicitario, por ello en Mobile Media Networks
se persigue generar valor para toda la cadena implicada, desde los
usuarios hasta las agencias de medios, pasando por los operadores
y los anunciantes. 

Incorporar publicidad a toda la cadena de comunicación, con lo que
se incrementa el acceso de los anunciantes a las comunicaciones, generando
servicios gratuitos y, lo más importante, reduciendo los costes para
el usuario.\textquotedblright{}\cite{Cap1.AdsMedia}


\subsection{Estructura organizacional}

En la Figura 1.1, se muestra la estructura organizacional de Mobile
Media Networks.

.

\vspace*{1cm}


\begin{center}
\begin{figure}[h]
\centering{}%%\includegraphics[scale=0.6]{\string"Images/Organigrama de la Empresa\string".png}\caption{Estructura organizacional MobileMedia Networks C.A.}
\end{figure}

\par\end{center}

La realización del proyecto de pasantía fue bajo el cargo de \textquoteleft{}Pasante\textquoteright{}
en la dirección de Tecnología de Mobile Media Networks.


\subsubsection{Resumen del Cargo de Pasante de Tecnología}

El pasante de Tecnología es el estudiante que busca conocimientos
que lo ayuden a desarrollarse como profesional. Las pasantías académicas
son las que representan un requisito para graduarse y se caracteriza
por tener tiempos académicos y proyectos específicos.\cite{Cap1.Mobile}
\newpage{}
