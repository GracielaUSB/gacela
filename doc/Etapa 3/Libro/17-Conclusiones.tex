
\chapter*{Conclusiones y Recomendaciones}

\selectlanguage{english}%
\addcontentsline{toc}{chapter}{Conclusiones y Recomendaciones}

\selectlanguage{spanish}%
\vspace{5mm}


La empresa Mobile Media Networks utiliza el sistema AdServer MobileX
como una plataforma publicitaria para la distribución de publicidad
en portales web y dispositvos móviles. Para la colocación simple y
apropiada en las aplicaciones, el AdServer MobileX se comunica con
un componente publicitario llamada AdView MobileX (o AdField, para
la versión de Blackberry). Este componente hace transparente al usuario
toda la lógica de negocios que está implementada, ofreciendo al AdServer
datos importantes para la segmentación y el reporte, además de reducir
la carga del mismo.

El presente proyecto consistió en el diseño de un protocolo de comunicación
simple y compacto, y el diseño y la implementación de un componente
que cumple con los requerimientos establecidos al comienzo de la pasantía.
Estos requerimientos se dividieron en tres categorías: Requerimientos
Básicos, Requerimientos Avanzados y Requerimientos Adicionales.

A partir de los requerimientos básicos se implementó un componente
que tiene la capacidad de construir una petición a un servidor, realizar
la comunicación e interpretar la respuesta. A partir de esta respuesta
coloca la creatividad en el lugar y de la manera adecuada, soportando
todas las formas publicitarias que maneja la empresa y sin afectar
de ninguna manera la interfaz de usuario.

Los requerimientos avanzados aportaron al componente un poco más de
inteligencia, siendo capaz de decidir cuando pedir una publicidad
y cuando aprovechar lo máximo una que ya se descargó. Además, las
creatividades se cargan inmediatamente al momento de abrir la aplicación,
haciendo que el componente no deje espacios inútiles en la interfaz.

Los últimos requerimientos aportaron al componente la posibilidad
de contribuir con información para las estadísticas y reportes que
genera el servidor, además de aportar la lógica necesaria para mostrar
la publicidad a pesar de no tener conexión de red.

La puesta en producción de los módulos del servidor que permitan realizar
la comunicación con el nuevo protocolo, o que permitan obtener nueva
información que envía el componente no están en el alcance del proyecto.
Esto obedece a una decisión que debe tomar la gerencia de la empresa
en conjunto con los miembros de cargos más altos.

La metodología AUP contribuyó con la planificación del proyecto y
a la presentación de la información, la cual servirá a futuros desarrolladores
que se involucren con los dos componentes implementados.

La experiencia obtenida durante el desarrollo de este proyecto de
pasantía comprende nuevos elementos de aprendizaje vinculados tanto
a las nuevas tecnologías utilizadas como al desenvolvimiento laboral
y las nuevas relaciones construidas en la empresa.

Para terminar, se presentan algunas recomendaciones:
\begin{itemize}
\item Incorporación de funcionalidades para poder colocar elementos de contenido
interactivo Rich Media.
\item Incorporación de una funcionalidad que consista en que cuando el usuario
haga clic en una creatividad que promociona un servicio de otra aplicación,
chequee si la apliación existe, y de existir, lance la aplicación
en la sección donde se encuentra la venta del servicio.
\item Realizar la documentación de los nuevos protocolos y componentes,
pero tanbién de los anteriores y de las nuevas funcionalidades que
se incorporarán al servidor. El sistema MobileX en conjunto es muy
complejo, con clientes publicitarios escritos para distintas plataformas
y un servidor muy amplio con muchos servicios ofrecidos.
\end{itemize}
\begin{onehalfspace}
\newpage{}\end{onehalfspace}
