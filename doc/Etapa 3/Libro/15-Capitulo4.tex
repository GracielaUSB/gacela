
\chapter{Marco Tecnológico}

\thispagestyle{empty} 

A continuación, se presentan los aspectos tecnológicos para el desarrollo
del proyecto de pasantía, se describen brevemente y se indican su
uso como parte de la solución propuesta. 


\section{Android}

Es un sistema operativo para dispositivos móviles tanto teléfonos
celulares como tabletas. Fue desarrollado por Android Inc, firma que
fue comprada por Google en el año 2005. Es el más utilizado en el
mundo, esto a razón del bajo costo que ofrece a las empresas fabricantes
de los dispositivos. Además Android también es personalizable y ligero,
cualidades muy importantes y atractivas. La mayoría del código de
Android está liberado bajo la licencia Apache.

El sistema operativo Android consta de un núcleo basado en los núcleos
de Linux 2.6 y 3.x, con librerías, APIs y middleware escritos en C.
El software de aplicación se ejecuta sobre un framework que incluye
librerías compatibles con Java basadas en Apache Harmony. Usa Dalvik
como máquina virtual, compilando las aplicaciones en tiempo de ejecución.Android
usa OpenCore como Framework, SQLite como manejador de base de datos,
WebKit como motor de renderizado, OpenGL como interfaz de programación
de API gráfica, un administrador de interfaz gráfica, un motor gráfico
SGL, una biblioteca estándar de C llamada Bionic y una capa de conexión
segura (SSL).\cite{Cap4.Android}

La última versión es la 4.2.1, llamada Jelly Bean (correspondiente
al API 17). Salió al mercado en noviembre del año 2012.

Las aplicaciones, tanto pagas como gratuitas, pueden ser descargadas
de la tienda virtual Play Store, mediante una aplicación con el mismo
nombre.


\subsection{Java}

Es un lenguaje de programación de alto nivel orientado a objetos desarrollado
por Sun Microsystems. Las aplicaciones escritas en Java son compiladas
a bytecode y pueden ser ejecutadas en cualquier máquina virtual de
Java. Gracias a su portabilidad, Java es usado para el desarrollo
de aplicaciones móviles para las plataformas Blackberry y Android.

Android ha creado un conjunto de librerías en este lenguaje que, en
conjunto con las librerías propias de Java, permiten el desarrollo
de aplicaciones para dispositivos portátiles que tengan instalado
el sistema operativo Android. En el portal web http://developer.android.com/reference/packages.html
se puede encontrar la referencia de las librerías que soporta una
aplicación Android, con la facilidad de filtrarlas por nivel de API.

Además de las librerías propias de Java, y las agregadas por Android,
se utilizaron librerías de terceros para poder incluir algunas funcionalidades
que no poseen las versiones de los sistemas operativos que el componente
debe soportar.


\section{iOS}

Es un sistema operativo para dispositivos móviles desarrollado y distribuido
por la compañía Apple, para ser usado únicamente en los dispositivos
fabricados por la misma firma. Entre estos dispositivos se encuentran
el iPhone, iPod Touch, iPad y el Apple TV. 

iOS es un derivado de OS X, el cual es un sistema operativo tipo Unix.
La estructura de iOS está basado en una abstracción de cuatro capas:
la Capa de Núcleo del Sistema Operativo, la Capa de Servicios, la
Capa Multimedia y la Capa de Cocoa Touch (framework de interfaz de
usuario). Está escrito en C, C++ y Objective-C. 

La última versión es la 6.1, disponible desde diciembre del 2012.

Las aplicaciones, tanto pagas como gratuitas, pueden ser descargadas
de la tienda virtual App Store, mediante una aplicación con el mismo
nombre.


\subsection{Objective-C}

Es un lenguaje de programación orientado a objetos de alto nivel.
Su sintaxis es similar a la de Smalltalk. Es usado para el desarrollo
de los sistemas operativos de Apple OS X y iOS, así como para las
Interfaces de Programación de Aplicaciones (APIs) Cocoa y Cocoa Touch.

Solo se permite el desarrollo de aplicaciones en Objective-C mediante
el uso de un IDE llamado Xcode, el cual se puede descargar gratuitamente
si se posee el hardware y software de Apple. La última versión de
Xcode es la 4.5.2. Xcode permite instalar los simuladores de los dispositivos
de las últimas versiones para la realización de las pruebas. La documentación
sobre las librerías para el desarrollo de aplicaciones se encuentra
en el portal web ttps://developer.apple.com/library/ios/navigation/index.html. 


\section{RPC}

La Llamada a Procedimiento Remoto (Remote Procedure Call) es un mecanismo
para las aplicaciones Cliente-Servidor, desarrollado por Sun Microsystems.
En la comunicación participa un procedimiento invocador (cliente)
y un procedimiento invocado (servidor). El objetivo de RPC es que
el cliente realice una llamada a un procedimiento del servidor como
si el mismo sea local. 

En lineas generales, para realizar la llamada a un procedimiento remoto,
el programa cliente debe enlazarse con un procedimiento de librería
llamado \textit{stub} del cliente, que representa el procedimiento
del servidor en el espacio del cliente. De la misma forma, el servidor
se enlaza con un procedimiento llamado \textit{stub} del servidor.
Ambos procedimiento ocultan el hecho de que la llamada no es local. 

La comunicación sconsta de los siguientes pasos:
\begin{enumerate}
\item El cliente llama a \textit{stub} del cliente. Esta es una llamada
local, por lo que los parámetros se colocan en la pila.
\item El \textit{stub }del cliente empaqueta los parámetros en un mensaje
y realiza una llamada de sistema para enviar el mensaje.
\item El kernel envía el mensaje desde la máquina cliente a la máquina servidor.
\item El stub del cliente llama al procedimiento servidor con los parámetros.
\end{enumerate}
La respuesta del servidor al cliente se realiza de la misma forma.
\cite{Cap4.RPC1}


\subsection{XML-RPC }

Es una llamada a procedimiento remoto que usa HTTP como protocolo
de transporte y XML para la codificación de los mensajes. Permite
la transmisión de estructuras de datos complejas, pero la interpretación
de las mismas es sencilla. \cite{Cap4.XMLRPC}

XML-RPC es la forma de comunicación que tiene actualmente el AdServer
MobileX con el componente AdField MobileX. Tanto la petición que hace
el cliente como la respuesta que recibe son empaquetadas en un XML,
y ambas contienen muchos etiquetas que son inútiles tanto para el
cliente como para el servidor.


\subsubsection{XML (eXtensible Markup Language) }

Es un lenguaje desarrollado por el World Wide Web Consortium (W3C)
que es usado para el intercambio de información entre distintas plataformas.
Su flexibilidad permite que sea usado para representar cualquier estructura
de datos, desde arreglos y árboles hasta, por ejemplo, un servicio
web o una base de datos. 


\subsection{JSON-RPC}

Es un protocolo de llamada a procedimiento remoto similar a XML-RPC,
solo que el formato de codificación es JSON en lugar de XML. Utilizar
JSON permite enviar estructuras de datos tan complejas como las que
se pueden enviar con XML-RPC, pero el mensaje es más ligero. \cite{Cap4.JSONRPC}

Una variación del JSON-RPC es la forma de comunicación propuesta para
la comunicación entre el AdServer MobileX y el nuevo componente AdView
MobileX. Los argumentos de la solicitud de publicidad no se empaquetarán
en un JSON, si no que se enviarán como argumentos al hacer la petición
HTTP. La respuesta será un JSON que contenga solo los campos necesarios
para mostrar y reportar la publicidad.


\subsubsection{JSON (JavaScript Object Notation) }

Es un lenguaje que ofrece un formato ligero para el intercambio de
información, siendo una alternativa a XML. Su ventaja frente a XML
es que su formato permite la creación de un diccionario, por lo que
no son necesarias el exceso de banderas que necesita XML para la represantación
de las estructuras. En la figura 4.1 se ven los dos formatos representando
los mismo datos.

\begin{center}
\begin{figure}[h]
\centering{}%%\includegraphics[scale=0.6]{Images/xmlvsjson}\caption{Representación de datos con elos formatos XML y JSON.}
\end{figure}

\par\end{center}

\newpage{}
