
\chapter{Marco Metodológico}

\thispagestyle{empty} 

Este capítulo presenta los aspectos metodológicos para el desarrollo
del proyecto de pasantía, se realiza una breve descripción la metodología
utilizada así como su implementación en este proyecto.


\section{AUP}

El Proceso Unificado Agil (\textit{Agile Unified Process}, AUP o AgileUP)
es una metodología para el desarrollo de software, basado en el Rational
Unified Process (RUP) de IBM. Describe de una forma simple la forma
de desarrollar aplicaciones de software de negocio utilizando técnicas
ágiles y conceptos que aún se mantienen válidos en RUP. \cite{Cap3.AUP2}

AUP emplea técnicas ágiles, incluyendo Desarrollo Guiado por Pruebas
(\textit{Test Driven Development} o TDD), modelado ágil, gestión ágil
de cambios y refactorización de bases de datos. Algunas de estas técnicas
formaron parte integral del desarrollo del proyecto. 

AUP establece metas y objetivos donde las colaboraciones son usadas
para lograr resultados. Una colaboración puede referirse a un trabajo
realizado por un sólo individuo que interactúa con otros, o puede
referirse a un grupo de trabajo que involucra a múltiples individuos
que interactúan unos con otros y con agentes externos. \cite{Cap3.AUP1} 

Para el desarrollo de este proyecto la metodología AUP fue implementada
siguiendo la documentación de la versión 1.1 del Proceso Unificado
Ágil. Se emplea AUP debido a que presenta un equilibrio entre las
etapas de diseño, planificación y documentación del sistema, con la
implementación del mismo, por lo tanto, se sigue manteniendo un desarrollo
estructurado y coherente, pero brindando una mayor flexibilidad para
el programador, tanto para en tiempo efectivo de implementación, como
en el manejo de cambios en el sistema a través del proceso de desarrollo.


\section{Fases de AUP}

La metodología AUP, (al igual que RUP), busca dividir el desarrollo
de software en cuatro (4) fases, cada una de las cuales comprende
técnicas y disciplinas de desarrollo, que permiten iterar sobre ellas
hasta lograr el cumplimiento de los objetivos planteados. Al ser una
metodología flexible y ágil, AUP permite que las técnicas que componen
cada fase, se puedan adecuar a las necesidades y realidades del proyecto. 

A continuación, se muestra un gráfico con las fases de la metodología
AUP (Figura 3.1).

\begin{center}
\begin{figure}[h]
\centering{}%%\includegraphics[scale=0.65]{\string"Images/Fases AUP\string".png}\caption{Fases y disciplinas de la metodología AUP \cite{Cap3.AUPFases}.}
\end{figure}

\par\end{center}

El eje vertical representa la dimensión dinámica, relativa al tiempo
y el eje horizontal la dimensión estática.


\section{Implementacion en el Proyecto de Pasantia}

AUP es una metodología de cuatro fases, cada una compuesta por un
conjunto de técnicas y disciplinas de desarrollo. Por ser una metodología
ágil, estas técnicas y disciplinas pueden ser adaptadas a las necesidades,
recursos e intereses del proyecto y de la empresa.

Durante el desarrollo de este proyecto se llevaron a cabo las cuatro
fases de la metodología: inicio, elaboración, construcción y transición.
Las fases de inicio y elaboración se llevaron a cabo en una iteración,
la fase de construcción en tres iteraciones por cada plataforma y
la fase de transición en una iteración. 


\subsection{Inicio (\textit{Inception})}

Esta es la primera fase del ciclo de vida donde se identifica el alcance
inicial del proyecto. El objetivo principal es archivar el consenso
de los interesados del proyecto en relación a los objetivos del mismo
para obtener financiamiento. \cite{Cap3.AUPFases}

En el contexto de este proyecto los objetivos dispuestos para esta
fase están centrados en la definición de los requerimientos del sistema.
Las actividades realizadas comprenden la introducción a la empresa,
investigación acerca de la situación actual, obtención y modelación
inicial de los requerimientos, planteamiento y estudio de las herramientas
a utilizar, elaboración de un plan de riesgos, elaboración inicial
de un plan de pruebas, elaboración inicial del glosario y la planificación
de actividades para las siguientes iteraciones. 


\subsection{Elaboración (\textit{Elaboration})}

La segunda fase consiste en probar la arquitectura potencial del sistema.
Se trata de determinar que el equipo pueda desarrollar un sistema
que satisfaga los requisitos, por lo que se realiza una construcción
de extremo a extremo del esqueleto del trabajo, conocido como \textquotedblleft{}prototipo
de la arquitectura\textquotedblright{}. 

Los riesgos se detallan lo suficiente como para entender los riesgos
de la arquitectura y para asegurar que exista una comprensión de los
alcances de cada requerimiento para que la planificación posterior
se puede llevar a cabo. {[}19{]}

En este proyecto las actividades comprendidas durante esta fase involucran
la identificación de los riesgos técnicos, el modelado de la arquitectura
y diseños de prototipos, así como la refinación de documentos desarrollados
en la fase anterior. 


\subsection{Construcción (\textit{Construction})}

Esta fase involucra la construcción de un software funcional sobre
una base regular e incremental, las cuales cumplan con las prioridades
más importantes para los involucrados o usuarios del proyecto. El
objetivo de esta fase consiste en desarrollar el sistema hasta el
punto que se encuentra listo para la pre-producción de pruebas. {[}19{]} 

Para este proyecto esta fase presenta tres iteraciones: 
\begin{itemize}
\item Primera iteración:

\begin{itemize}
\item Diseño de un nuevo protocolo de comunicacion con el servidor.
\item Diseño e implementación del mecanismo de obtención de datos.
\end{itemize}
\item Segunda iteración:

\begin{itemize}
\item Diseño e implementación de almacenamiento en cache.
\item Diseño e implementación del funcionamiento al no tener red de datos.
\end{itemize}
\item Tercera iteración:

\begin{itemize}
\item Diseño e implementación de la lógica de retraso de peticiones.
\end{itemize}
\end{itemize}

\subsection{Transición (\textit{Transition})}

Esta última fase consiste en la validación y despliegue del sistema
en su ambiente de producción. Durante esta fase deben hacerse pruebas
extensivas, para asegurarse de que el sistema puede ser desplegado
de manera segura y eficiente. {[}19{]}

En el desarrollo de este proyecto en la fase de transición se realizaron
las pruebas necesarias para comprobar el funcionamiento correcto de
los componentes para las dos plataformas y la refinación de documentos.
El alcance de este proyecto de pasantía sólo abarca hasta la generación
de documentos, con la elaboración del informe de pasantía. En este
sentido el proceso de puesta en producción queda fuera de las 20 semanas
estipuladas. 

\newpage{}
