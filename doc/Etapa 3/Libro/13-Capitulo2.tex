
\chapter{Marco Teórico}

\thispagestyle{empty} 

A continuación, se presentan los conceptos teóricos relevantes sobre
los cuales se fundamentó el proyecto de pasantía, que incluyen el
marco de trabajo y las herramientas utilizadas.


\section{Conceptos Básicos}


\subsection{Mobile X}

Es una herramienta para la gestión de campañas publicitarias en dispositivos
móviles basado en OpenX. MobileX cumple con el modelo cliente-servidor.
El cliente es el componente encargado de mostrar la publicidad en
un portal web o en una aplicación, mientras que el servidor es el
responsable de proveer al cliente la publicidad adecuada, la cual
se identifica mediante un número llamado identificador de zona (zoneId). 

La comunicación entre ambos se realiza mediante el protocolo XML-RPC
para la versión 1.0 del cliente, y el protocolo JSON-RPC para la versión
2.0.


\subsubsection{AdServer MobileX}

Es una plataforma publicitaria para la entrega de publicidad en dispositivos
móviles, basado en el AdServer de OpenX. Posee características para
el manejo y la distribución de campañas y un sistema de rastreo para
presentar las estadísticas.

Está licenciado bajo GNU General Public License y escrito en su mayor
parte en PHP, utilizando MySQL como manejador de base de datos. La
instalación esta hecha sobre los servidores de Amazon, bajo el servicio
web de computación de nube (Cloud Computing) Amazon EC2 (Elastic Cloud
Computing).

El AdServer maneja varios módulos para la entrega y el reporte de
las campañas. Una campaña es un conjunto específico de banners, los
cuales tienen una fecha de inicio y de finalización, así como una
frecuencia con la que serán mostrados. Se manejan tres tipos de campaña:
\begin{enumerate}
\item Exclusiva: es aquella campaña que se ejecuta antes que cualquier otra. 
\item De contrato: es aquella campaña que será entregada un número específico
de veces por días, por un número específico de días. 
\item Remanente: es aquella campana que se mostrará luego de las anteriores,
limitada por un número de impresiones o una fecha de finalización. 
\end{enumerate}
Un banner es cualquier tipo de contenido creativo que será mostrado
como una publicidad. Puede ser de formato JPG, PNG, GIF y SWF, así
como ser escrito en Javascript, HTML. Un banner para una campaña puede
tener varias versiones por los distintos tamaños predeterminados de
los dispositivos actuales para los cuales será distribuida la publicidad.


\subsubsection{AdField MobileX}

Es un componente publicitario que actúa como cliente para la plataforma
Blackberry cuya finalidad es mostrar las campañas publicitarias en
los dispositivos, así como reportar la impresión y el clic de las
mismas. Esta escrito completamente en java, utilizando algunas librerías
de RIM. Su nombre se debe a que en Blackberry un área rectangular
se denomina un campo (field), por lo que una creatividad se adapta
a esta definición.

Actualmente se encuentra operativa la versión 1.0 del componente y
en desarrollo la versión 2.0 en base a las nuevas funcionalidades
y cambios propuestos en este proyecto de pasantía. 

Para poder desplegar el contenido publicitario, el cliente debe realizar
una solicitud al servidor enviando un número de zona. La zona identifica
un espacio donde las formas creativas son desplegadas. Cuando el servidor
recibe el número de zona, ejecuta un código de invocación, el cual
retorna la información del banner necesaria para ser mostrado.

La información mínima que se debe tener para poder desplegar el banner
es:
\begin{enumerate}
\item Dirección de contenido: dirección url donde se encuentra la creatividad. 
\item Dirección de impresión: dirección url que será llamada cuando se muestre
la creatividad en la aplicación.
\item Dirección de clic: dirección url que será llamada cuando se haga clic
sobre la creatividad en la aplicación. 
\item Ancho: ancho que mide la creatividad. (Debe ser menor que el ancho
de la pantalla del dispositivo). 
\item Alto : alto que mide la creatividad. (Debe ser menor que el alto de
la pantalla del dispositivo). 
\item Identificador de banner: identificador del tipo de creatividad. 
\item Tipo de contenido: define si la creatividad es de formato jpg, png,
gif, entre otros posibles.
\item Tipo de campaña: define si la campaña es exclusiva, de contrato o
remanente.
\end{enumerate}

\subsubsection{AdView MobileX}

Es un componente publicitario que actúa como cliente para las plataformas
iOS y Android. Este componente cumple con las mismas funcionalidades
que el componente AdField MobileX, solo que adopta este nombre porque
un área rectangular en estas plataformas se denomina una vista (view),
término bajo el cual se adapta la definición de una creatividad. 

La versión de Android está escrito completamente en Java, utilizando
algunas librerías de propias de Android, mientras que la versión de
iOS está escrito en Objective-C. Para ambas versiones fue necesaria
la utilización de algunas librerías de terceros para implementar algunas
funcionalidades.

Actualmente se encuentra operativa la versión 2.0 del componente,
incluyendo los cambios pertinentes del servidor.

Para esta versión el cliente debe realizar una solicitud al servidor
enviando un número de zona, el tamaño de la pantalla (ancho y alto),
un identificador del registro (cookie), el Código de Area de País
(Mobile country code) y el Código de Area del Operador de Telefonia
(Mobile network code).


\section{Herramientas}


\subsection{Modelo Cliente - Servidor}

La arquitectura cliente-servidor permite al usuario en un dispositivo
movil, llamado el cliente, requerir algún servicio de una máquina
a la que está unido, llamado el servidor, mediante una red, ya sea
de datos o inalamrbica (Wi-Fi). Estos servicios pueden ser peticiones
de datos de una base de datos, de información contenida en archivos
o los archivos en sí mismos, o peticiones de imprimir datos en una
impresora asociada~\cite{Cap2.ClienteServidor}.

El modelo cliente-servidor se fundamenta en la idea de que un proveedor
de datos por si mismo puede abastecer las solicitudes de numerosos
clientes que requieren de sus servicios.

El componente es un ejemplo de un sistema basado en este modelo. Cada
cliente estará representado por el componente publicitario, que realizará
solicitudes al servidor en el momento oportuno solicitando una creatividad;
y el servidor, que contiene la aplicación de gestión de solicitudes
de publicidad, atenderá cada una de las solicitudes del cliente.


\subsection{Agente de Usuario}

El agente de usuario es una cadena de caracteres que identifica a
un dispositivo o un portal web. Los datos que incluye son:
\begin{enumerate}
\item El nombre de la aplicación y la versión.
\item El tipo de explorador y su versión.
\item El sistema operativo.
\item Las extensiones instaladas en el sistema o en el explorador.
\end{enumerate}
\begin{figure}[h]
%%\includegraphics{Images/useragent}

\caption{Formato del agente de usuario \cite{Cap2.Useragent}}
\end{figure}



\subsection{Librerías de Terceros}

Las librerías de terceros (Third-party software component) son componentes
desarrollados para implementar alguna funcionalidad que no esté presente
en la plataforma que se trabaja o para mejorar las que existan. Pueden
ser vendidas o distribuidas libremente, por lo que cada librería posee
una licencia asociada límita su uso. La ventaja de estas librerías
es que el desarrollador no tiene que estar escribiendo un nuevo componente,
pero tiene que estar conciente de que existe la posibilidad de que
el componente presente errores que el creador original no manejó.
